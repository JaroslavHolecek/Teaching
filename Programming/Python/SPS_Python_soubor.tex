\section{Práce se souborem}
Často chceme, aby si náš program uchoval informaci i po svém ukončení (např. dosažený výsledek ve hře, uživatelské nastavení). Pokud těchto informací není mnoho, můžeme je uložit do souboru. Do souboru je můžeme uložit i když jich mnoho je, ale v takovém případě je lepší využít databázi, která bude s uloženými daty pracovat rychleji. Někdy také přímo chceme, aby náš program vytvořil soubor - například seznam uiživatelů či výsledkovou listinu.\\

\subsection{Typy souborů}
Pokud pomineme specializované typy souborů, jako jsou obrázky (jpg, png, \dots) či dokumenty (pfd, xml, xlsx, doc, \dots) umí samotný Python pracovat především s dvěma typy souborů:
\begin{enumerate}
\item \textbf{Textový}, ve kterém je uložen text - tedy znaky ASCII, UTF8, cp1250, případně dalších kódování.\\ Tento typ souboru je defaultní, tedy pokud neurčíme jinak, použije se právě textový soubor. 
\item \textbf{Binární}, ve kterém jsou uložená \uv{surová} data - tedy 1 a 0. Napřiklad obrázky, videa, zvukové soubory.
\end{enumerate}
Samozřejmě i v textovém soubou jsou uloženy jen 1 a 0 (stejně jako všude v počítači). Zde jsou seřazeny v přesně daném uspořádání (určeném kodováním) a tedy je takový soubor čitelný pro jakýkoliv jiný program, který umí text s daným kódováním číst - např. SublimeText, Poznámkový blok, Gedit apd.

\subsection{Textový soubor}
Pokud chceme pracovat se souborem, musíme tento \textbf{soubor} nejprve \textbf{otevřít}. To provedeme funkcí \uv{\textbf{open()}}. Soubor můžeme otevřít v různých režimech - pro čtení, pro zápis, pro obojí. Podle zvoleného režimu můžeme ze souboru načítat data (funkce \textbf{read()}, \textbf{readline()}), zapisovat data (funkce \textbf{write()}), nebo obojí.\\

\subsubsection{Otevření souboru - open()}
Před začátkem práce se souborem jej musíme nejprve otevřít. Při otevírání souboru může nastat chyba (viz. níže, nebo nemáme práva pro přístup do složky apd.). Takovou chybu je vhodné v programu ošetřit (viz. kód \ref{code:python_try}).\\
K otevření souboru využijeme funkci \textbf{open()}, které musíme předat argumenty:
\begin{itemize}
\item[\textbf{název souboru}] , kterým určíme s jakým souborem chceme pracovat.\\Pokud zadáme celou cestu k souboru ($C:\backslash\backslash muj\backslash adresar\backslash se\backslash soubory\backslash muj_prvni_soubor.txt$), můžeme otevřít soubor kdekoliv. Pokud zadáme pouze název souboru ($muj_prvni_soubor.txt$), otevře se tento soubor ve stejné složce, ve které máme náš program (skript).
\item[\textbf{režim (mód)}] , kterým určime, zda chceme ze souboru číst, chceme do něj zapisovat, nebo obojí. Defaultní režim (tedy takový, který se použije, pokud žádný nezadáme je: čtení textového souboru). Režim změníme pomocí následujících hodnot:
	\begin{itemize}
	\item[Režim]
		\begin{itemize}
		\item[r] čtení\\pokud soubor neeexistuje, nastane chyba
		\item[a] přidání na konec - ponechá současný obsah souboru a nově zapsaný připíše na konec\\pokud soubor neexistuje, vytvoří se
		\item[w] přepsání souboru - smaže současný obsah souboru a začne psát od začátku\\pokud soubor neexistuje, vytvoří se
		\item[x] vytvoření souboru\\pokud soubor existuje, nastane chyba
		\item[+] doplnění druhé možnosti ("r+" je čtení a zápis, "w+" je zápis a čtení)
		\end{itemize}
	\item[Typ]
		\begin{itemize}
		\item[t] textový
		\item[b] binární
		\end{itemize}
	\end{itemize}
\end{itemize}

Funkce \uv{open()} ovetře soubor, ale také nám \textbf{vrátí objekt}, ve kterém je \textbf{uloženo spoustu užitečných informací} - zda se soubor podařilo otevřít, v jakém režimu je soubor otevřený, v jakém místě v souboru právě čteme (případně do jakého místa právě zapisujeme). Tento objekt použijeme vždy, když budeme chtít se souborem něco provést.\\

Po ukončení práce se souborem je vhodné jej \textbf{zavřít} funkcí \uv{\textbf{close()}}.\\

\begin{minipage}[t]{.45\textwidth}
\begin{code}
\begin{minted}[linenos, escapeinside=!!]{python}
soubor = open("muj_prvni_soubor.txt", "r") !\label{scl:python_file_open}!
# zde muzeme cist ze souboru soubor
soubor.close() !\label{scl:python_file_close}!

with open("muj_prvni_soubor.txt", "r") as soubor: !\label{scl:python_file_with}!
	# zde muzeme cist ze souboru soubor
	# POZOR! Prikazy jsou odsazene - jsme uvnitr bloku with
\end{minted}

\captionof{listing}{Otevření souboru}
\label{code:python_soubor_open}
\end{code}
\end{minipage}
\begin{minipage}[t]{.45\textwidth}
\begin{enumerate}
\item[ř. \ref{scl:python_file_open}:] Otevření souboru pro čtení a vytvoření objektu \uv{soubor}.
\item[ř. \ref{scl:python_file_close}:] Uzavření souboru 
\vspace{2cm}
\item[ř. \ref{scl:python_file_with}:] Můžeme také využít blok \uv{with}, který za nás zavře soubor automaticky
\end{enumerate}
\end{minipage} 

\subsubsection{Čtení ze souboru}
Při čtení ze souboru se nám obsah souboru (nebo jeho část) načte do string proměnné v kódu. S tímto načteným obsahem poté pracujeme jako s jakýmkoliv jiným řetězcem (stringem).\\
Při čtení se v souboru posouvá ukazatel, který ukazuje na místo, kde jsme se čtením skončili (tento ukazatel je uložený v objektu, který nám vrátí funkce open() ). Pokud tedy přečteme 7 znaků a dále v programu ve čtení pokračujeme, budou se znaky načítat od místa, kde jsme se čtením skončili (tedy od 8. znaku).
Při čtení souboru máme několik možností:
\paragraph{Přečtení celého souboru}
Pomocí funkce \uv{read()} přečteme celý soubor najednou.

\begin{minipage}[t]{.45\textwidth}
\begin{code}
\begin{minted}[linenos, escapeinside=!!]{python}
with open("muj_prvni_soubor.txt", "r") as soubor: 
	obsah_souboru = soubor.read() !\label{scl:python_file_read}!
	print(obsah_souboru) !\label{scl:python_file_read_print}!
\end{minted}

\captionof{listing}{Přečtení celého souboru}
\label{code:python_file_read}
\end{code}
\end{minipage}
\begin{minipage}[t]{.45\textwidth}
\begin{enumerate}
\vspace{0.5cm}
\item[ř. \ref{scl:python_file_read}:] Uložení celého obsahu souboru do proměnné \uv{obsah\_souboru}
\item[ř. \ref{scl:python_file_read_print}:] Vypsání do konzole 
\end{enumerate}
\end{minipage}

\paragraph{Přečtení daného počtu znaků}
Pomocí funkce \uv{read()} můžeme také přečíst pouze zadaný počet znaků.

\begin{minipage}[t]{.45\textwidth}
\begin{code}
\begin{minted}[linenos, escapeinside=!!]{python}
with open("muj_prvni_soubor.txt", "r") as soubor: 
	sedm_znaku = soubor.read(7) !\label{scl:python_file_read_arg}!
	print(sedm_znaku)
\end{minted}

\captionof{listing}{Přečtení zadaného počtu znaků}
\label{code:python_file_read_arg}
\end{code}
\end{minipage}
\begin{minipage}[t]{.45\textwidth}
\begin{enumerate}
\vspace{0.5cm}
\item[ř. \ref{scl:python_file_read_arg}:] Načtení pouze 7 znaků
\end{enumerate}
\end{minipage}

\paragraph{Přečtení daného počtu znaků}
Pomocí funkce \uv{readline()} přečteme jeden řádek v souboru.

\begin{minipage}[t]{.45\textwidth}
\begin{code}
\begin{minted}[linenos, escapeinside=!!]{python}
with open("muj_prvni_soubor.txt", "r") as soubor: 
	radek = soubor.readline() !\label{scl:python_file_readline}!
	print(radek)
\end{minted}

\captionof{listing}{Přečtení jednoho řádku}
\label{code:python_file_readline}
\end{code}
\end{minipage}
\begin{minipage}[t]{.45\textwidth}
\begin{enumerate}
\vspace{0.5cm}
\item[ř. \ref{scl:python_file_readline}:] Načtení jednoho řádku
\end{enumerate}
\end{minipage}

\paragraph{Přečtení celého souboru po řádcích}
Velmi často chceme přečíst celý soubor po jednotlivých řádcích. V Pythonu k tomu můžeme využít jednoduchou konstrukci za použití \uv{for}. For-cyklus slouží k procházení libovolného seznamu prvek po prvku - a v Pythonu je soubor také seznam - seznam řádků.

\begin{minipage}[t]{.45\textwidth}
\begin{code}
\begin{minted}[linenos, escapeinside=!!]{python}
with open("muj_prvni_soubor.txt", "r") as soubor: 

	for radek in soubor:  !\label{scl:python_file_for}!
		print(radek)
\end{minted}

\captionof{listing}{Přečtení celého souboru po řádcích}
\label{code:python_file_for}
\end{code}
\end{minipage}
\begin{minipage}[t]{.45\textwidth}
\begin{enumerate}
\vspace{0.5cm}
\item[ř. \ref{scl:python_file_for}:] Tato konstrukce projde postupně celý soubor - do proměnné \uv{radek} bude v káždém průchodu cyklem uložen jeden řádek souboru
\end{enumerate}
\end{minipage}

\subsubsection{Zápis do souboru}
Zápis do souboru je velmi přímočarý. Po otevření souboru ve správném režimu, použijeme funkci \uv{write()}, která umí zapsat string do souboru.

\begin{minipage}[t]{.45\textwidth}
\begin{code}
\begin{minted}[linenos, escapeinside=!!]{python}
with open("muj_prvni_soubor.txt", "w") as soubor:  !\label{scl:python_file_write_open}!
	soubor.write("Toto je můj první zápis do souboru.") !\label{scl:python_file_write}!
	soubor.write(str(5)) !\label{scl:python_file_write_str}!
\end{minted}

\captionof{listing}{Zápis do souboru}
\label{code:python_file_write}
\end{code}
\end{minipage}
\begin{minipage}[t]{.45\textwidth}
\begin{enumerate}
\vspace{1cm}
\item[ř. \ref{scl:python_file_write_open}:] Při otevírání souboru zvolíme správný režim.
\item[ř. \ref{scl:python_file_write}:] Zápis textu do souboru.
\item[ř. \ref{scl:python_file_write_str}:] Pokud chceme zapsat číslo, musíme ho nejprve převést na string.
\end{enumerate}
\end{minipage}

\paragraph{Speciální znaky}
Obzvláště při práci se souborem využijeme speciální znaky:
\begin{itemize}
\item[\textbf{"$ \backslash t $"}] Tabulátor
\item[\textbf{"$ \backslash n $"}] Konec řádku
\end{itemize}


