\section{Formulář}
Formulář je speciální prvek, který nám umožní vložit na stránku políčka, do kterých můžeme (jako uživatelé stránek) zapsat libovolné údaje (Jméno, věk, datum, zaškrtnout jednu z více možností, zaškrtnout několik z více možností apd.).\\

\vspace{0.5cm}
HTML ani CSS nám neumožní s těmito zadanými hodnotami něco udělat - uložit si je, něco spočítat, \dots K tumu budeme potřebovat PHP (připadně JavaScript). O PHP si budeme povídat později. O JavaScriptu, pokud to stihneme, ještě později.\\

\subsection{Nastavení - <form>}
Formulář zapisujeme do tagu <form>.\\
V otevíracím tagu <form> nastavíme, co se má stát (na jakou stránku má přejít prohlížeč) po kliknutí na odesílací tlačítko - to nastavíme pomocí atributu \uv{action}\\
Dále také nastavíme, jakou metodou se mají odeslat data mezi stránkami - to nastavíme pomocí atributu \uv{method}.\\
Oba tyto atributy pro nás zatím nejsou důležité, neboť nemáme žádný skript (program), který by data zpracoval.

\subsection{Vstupní políčka - <input> a další}
Velké množství typů vstupních políček naleznete v dokumentaci, případně zde: \url{https://www.w3schools.com/html/html_form_input_types.asp	}
Vstupní políčka jsou jednotlivé prvky formuláře. Těchto prvků je velké množství a zde si ukážeme jen některé základní.\\
Růžné vstupní typy políček nastavíme nejčastěji pomocí atributu \uv{type}, ale několik málo jich má vlastní tag.\\
Dále je vhodné (a později nutné) přiřadit každému políčku \uv{name}, pomocí kterého z něj později získáme data.\\
A v neposlední řadě zde užijeme atribut \uv{id}. Pomocí něj můžeme ke vstupnímu políčku přiřadit nápis (label), který pak bude reagovat na naše kliknutí.

\subsubsection{Tlačítko - <button>}
Tlačítko, na které můžeme kliknout a pomocí atributu \uv{onclick} nastavit, co se poté stane.

\subsubsection{Dlouhý text - <textarea>}
Pole pro zapsání delšího textu. Pomocí atributů \uv{rows} a \uv{cols} nastavíme velikost pole.

\subsubsection{Rozbalovací seznam - <select>}
Rozbalovací seznam, ze kterého můžeme vybrat jednu, nebo více možností (podle atributu \uv{multiple}).\\
Jednotlivé možnosti píšeme do tagů <option>.

\subsubsection{Tlačítko - submit}
Tlačítko, po jehož zmáčknutí se odešlou data formuláře na určenou stránku.
\subsubsection{Text - text}
Textové pole vhodné např. pro jméno.
\subsubsection{Heslo - password}
Pole, ve kterém není viditelný obsah, který do něj zapíšeme (místo všech znaků jsou např. černé tečky)
\subsubsection{Email - email}
Pole, ve kterém prohlížeč pohlída, zda je zadaná emailová adresa.
\subsubsection{Zaškrtávací políčka - checkbox}
Umožňuje zaškrtnout (a odškrtnout) několik nabízených možností
\subsubsection{Číslo - number}
Prohlížeč pohlídá, zda je zadáno číslo
\subsubsection{Soubor - file}
Otevře se okno pro výběr souboru z počítače
\subsubsection{Spoustu a spoustu dalších...}
Barva, datum, telefoní číslo, \dots 


\begin{minipage}[t]{.45\textwidth}
\begin{code}
\begin{minted}[linenos, escapeinside=||]{html}
<!DOCTYPE html>
<html>
 <body>
  
  <form action="zpracovani_dat.php" method="post"> |\label{scl:html_form_start}|
   <input type="radio" id="muz" name="gender" value="muz"> |\label{scl:html_form_muz}|
   <label for="muz">Muž</label> |\label{scl:html_form_label_muz}|
   <br>   
   <input type="radio" id="zena" name="gender" value="zena"> |\label{scl:html_form_zena}|
   <label for="zena">Žena</label> |\label{scl:html_form_label_zena}|
   <br>
   <input type="radio" id="jiny" name="gender" value="jiny"> |\label{scl:html_form_x}|
   <label for="jiny">Jiný</label> |\label{scl:html_form_label_x}|
   <br>
   <br>
   <textarea name="message" rows="10" cols="30"> |\label{scl:html_form_textarea}|
     Napiš něco o sobě...
   </textarea> 
   <br>
   <br>
   <label for="em">Zadej email:</label>
   <input type="email" id="em" name="adresa"> |\label{scl:html_form_email}|
   <br>
   <input type="submit" value="Odeslat">
  </form> 

 </body>
</html>
\end{minted}

\captionof{listing}{Odkaz}
\label{code:html_odkaz}
\end{code}
\end{minipage}
\begin{minipage}[t]{.45\textwidth}
\begin{enumerate}
\vspace{9cm}
\item[ř. \ref{scl:html_form_start}:] V otevíracím tagu formuláře nastavíme další vlastnosti. 
\item[ř. \ref{scl:html_form_muz}, \ref{scl:html_form_zena}, \ref{scl:html_form_x}:] Vstupní políčka typu \uv{radio}.\\Všechny mají stejnou hodnotu atributu \uv{name} - tak prohlížeč pozná, že patří k sobě\\
Každý má své id - pomocí id k němu připojíme label.\\
Všechny mají \uv{value} - protože zde nic nevyplňujeme, je to hodnota, která se odešle ke zpracování.
\item[ř. \ref{scl:html_form_label_muz}, \ref{scl:html_form_label_zena}, \ref{scl:html_form_label_x}:] Popisky jednotlivých tlačítek\\
Pomocí atributu \uv{for} je můžeme projit s daným vstupním políčkem.
\item[ř. \ref{scl:html_form_textarea}:] Některé prvky mají samostatný tag.
\item[ř. \ref{scl:html_form_email}:] Některé prvky upravíme pomocí změny atributu \uv{type}.
\end{enumerate}
\end{minipage}\\







