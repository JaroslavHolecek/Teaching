\section{Formulář}
Pomocí PHP můžeme zpracovat data odeslaná přes formulář. Samotný formulář je vytvořený pomocí HTML.\\
Pro nastavení souboru, který bude zpracovávat data po jejich odeslání, nastavíme u formuláře atribut \uv{action}. Soubor, jehož název zapíšeme do attributu \uv{action} se spustí po klikutí na tlačítko ve formuláři - a bude mít k dispozici data zadaná do formuláře.\\
Můžeme spustit jakýkoliv soubor (uložený na serveru) - klidně znovu ten, ve kterém máme náš formulář. \\

\begin{minipage}[t]{.45\textwidth}
\begin{code}
\begin{minted}[linenos, escapeinside=!!]{html}
<html>
<body>

<form action="soubor_pro_zpracovani.php" method="post"> !\label{scl:action}!
Počet řádků: <input type="number" name="radky"><br>	!\label{scl:name_1}!
Počet sloupců: <input type="number" name="sloupce"><br> !\label{scl:name_2}!
<input type="submit" value="Zobraz tabulku">
</form>

</body>
</html> 
\end{minted}

\captionof{listing}{Formulář v HTML}
\label{code:php_form_html}
\end{code}
\end{minipage}
\begin{minipage}[t]{.45\textwidth}
\vspace{2.7cm}
\begin{enumerate}
\item[ř. \ref{scl:action}:] Atribut \uv{action} určuje, který soubor se spustí po odeslání dat\\
Atribut \uv{method} určuje, jakým způsobem se data odešlou
\item[ř. \ref{scl:name_1},\ref{scl:name_2}] Atribut \uv{name} musíme vyplnit, abychom mohli získat data z tohoto konkrétního políčka - určeného právě jeho \uv{name}.
\end{enumerate}
\end{minipage}

\subsection{Metoda GET}
Pokud pro odesílání dat zvolíme metodu \uv{GET}, budou přenášená data snadno čitelná pro kohokoliv. Jsou přenášena pomocí URL (můžete si například vyzkoušet vyhledat něco pomocí vašeho oblíbeného webového vyhledávače - po vyhledání se podívejte na adresní řádek vašeho prohlížeče.) Metoda \uv{GET} tedy není vhodná pro přenášení tajných informací, jako jsou hesla apd.\\
Po odeslání dat pomocí metody \uv{GET} budou data uložena v asociativním poli \uv{\$\_GET}. Toto pole je globální proměnná a můžeme jej tedy používat kdekoliv v kódu.\\
K informacím zadaným do formuláře se dostaneme pomocí  \uv{\$\_GET["nazev\_policka\_ve\_formulari"]}. \uv{nazev\_policka\_ve\_formulari} nastavíme políčku pomocí atributu \uv{name}.\\
Použití této metody nastavíme v atributu formuláře \uv{method}.\\

\begin{minipage}[t]{.45\textwidth}
\begin{code}
\begin{minted}[linenos, escapeinside=!!]{html+php}
<html>
<body>

<form action="soubor_pro_zpracovani.php" method="get"> !\label{scl:method_get}!
Počet řádků: <input type="number" name="radky"><br>
Počet sloupců: <input type="number" name="sloupce"><br>
<input type="submit" value="Zobraz tabulku">
</form>

<?php
$zadane_radky = $_GET["radky"];	!\label{scl:method_get_r}!
$zadane_sloupce = $_GET["sloupce"];	!\label{scl:method_get_s}!
?>

</body>
</html> 
\end{minted}

\captionof{listing}{Metoda GET}
\label{code:php_form_get}
\end{code}
\end{minipage}
\begin{minipage}[t]{.45\textwidth}
\vspace{2.7cm}
\begin{enumerate}
\item[ř. \ref{scl:method_get}:] Atribut \uv{method} nastavený na "get" určuje použití metody \uv{GET}.
\item[ř. \ref{scl:method_get_r}, \ref{scl:method_get_s}:] Po odeslání dat je můžeme použít pomocí \$\_GET[] a \uv{name} daného políčka.
\end{enumerate}
\end{minipage}


\subsection{Metoda POST} 
U metody \uv{POST} se data přenáší pomocí HTTP POST a nejsou tedy snadno čitelná pro kohokoliv. Navíc také můžeme poslat libovolný objem dat, což u metody \uv{GET} nelze (nicméně zde ve škole nás to příliš netrápí).\\
Po odeslání dat pomocí metody \uv{POST} budou data uložena v asociativním poli \uv{\$\_POST}. Toto pole je globální proměnná a můžeme jej tedy používat kdekoliv v kódu.\\
K informacím zadaným do formuláře se dostaneme pomocí  \uv{\$\_POST["nazev\_policka\_ve\_formulari"]}. \uv{nazev\_policka\_ve\_formulari} nastavíme políčku pomocí atributu \uv{name}. \\

\begin{minipage}[t]{.45\textwidth}
\begin{code}
\begin{minted}[linenos, escapeinside=!!]{html+php}
<html>
<body>

<form action="soubor_pro_zpracovani.php" method="post"> !\label{scl:method_post}!
Počet řádků: <input type="number" name="radky"><br>
Počet sloupců: <input type="number" name="sloupce"><br>
<input type="submit" value="Zobraz tabulku">
</form>

<?php
$zadane_radky = $_POST["radky"];	!\label{scl:method_post_r}!
$zadane_sloupce = $_POST["sloupce"];	!\label{scl:method_post_s}!
?>

</body>
</html> 
\end{minted}

\captionof{listing}{Metoda POST}
\label{code:php_form_post}
\end{code}
\end{minipage}
\begin{minipage}[t]{.45\textwidth}
\vspace{2.7cm}
\begin{enumerate}
\item[ř. \ref{scl:method_post}:] Atribut \uv{method} nastavený na "post" určuje použití metody \uv{POST}.
\item[ř. \ref{scl:method_post_r}, \ref{scl:method_post_s}:] Po odesání dat je můžeme použít pomocí \$\_POST[] a \uv{name} daného políčka.
\end{enumerate}
\end{minipage}



