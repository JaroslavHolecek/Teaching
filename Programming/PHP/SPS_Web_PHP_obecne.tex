\section{Obecné}
\subsection{Kód, který píše kód}
Kromě jiného budeme PHP využívat k tomu, abychom nemuseli ručně vypisovat dlouhý a opakující se HTML kód - třeba velkou tabulku.\\
Napíšeme si proto PHP skript, který po spuštění napíše tento HTML kód \uv{za nás}. Obvykle budeme pomocí PHP vytvářet pouze část HTML kódu a část napíšeme ručně.

\subsection{Jak to funguje}
Při použití PHP nám do hry vstupuje další prvek - a tím je server, který spouští naše PHP skripty. Obyčejný prohlížeč neumí PHP zpracovat.\\
Na této škole máme server Xeon na adrese xeon.spskladno.cz\\
Xeon má přístup k Vaší složce \uv{public\_html} na disku H: a v prohlížeči si můžete přes Xeon tuto složku zobrazit na adrese: xeon.spskladno.cz/~vas\_login (musíte na složce nastavit příslušná přístupová práva)\\
A jak to celé funguje?\\
\begin{enumerate}
\item PHP skript (soubor s naším php kódem) máme uložený na našem počítači (případně je uložený na serveru). 
\item Tento skript předáme serveru a ten zpracuje veškeré PHP příkazy. Zbylý obsah souboru (mimo bloky <?php \dots ?>) pouze opíše.
\item Z výsledků příkazů v php blocích a textu mimo tyto bloky vznikne na serveru nový soubor - tentokrát již obyčejný textový (obvykle html)
\item Výsledný textový (html) soubor server odešle do našeho prohlížeče, který nám ho zobrazí.
\end{enumerate}

\subsection{Rozdělení kódu}
PHP kód nemusíme mít v souboru pohromadě v jednom bloku \uv{<?php \dots ?>}. Takových bloků můžeme mít v souboru libovolné množství - a všechny \uv{se počítají} jako jeden kód.\\
Můžeme tedy zapsat do PHP jen ty části, které chceme měnit a ty, které zůstávají stále stejné mohou být pevně napsané v html.\\

\begin{minipage}[t]{.45\textwidth}
\begin{code}
\begin{minted}[linenos, escapeinside=!!]{html+php}
<html>
<body>

<?php
$default_value=15; !\label{scl:php_part1}!
?>

<form action="soubor_pro_zpracovani.php" method="post"> !\label{scl:request_method_post_set}!
Počet řádků: <input type="number" name="radky" value=<?php echo $default_value; ?><br> !\label{scl:php_part2}!
Počet sloupců: <input type="number" name="sloupce" value=<?php echo $default_value; ?>><br> !\label{scl:php_part3}!
<input type="submit" value="Zobraz tabulku">
</form>


</body>
</html> 
\end{minted}

\captionof{listing}{Rozdělení kódu}
\label{code:php_divide_code}
\end{code}
\end{minipage}
\begin{minipage}[t]{.45\textwidth}
\vspace{5cm}
\begin{enumerate}
\item[ř. \ref{scl:php_part1}, \ref{scl:php_part2}, \ref{scl:php_part3}:] Tři bloky php.\\Tyto bloky jsou součást jednoho kódu - tedy to, co uděláme v jednom bloku, můžeme využít v jiném (zde např. uložená proměnná)
\end{enumerate}
\end{minipage}

\subsection{Využití dat po obnovení stránky}
Pokud obnovím stránku (znovu načtu, refresh) ztratím veškeré proměnné (jejich hodnoty), které jsem doposud v kódu použil.\\
Při každém načtení stránky - tedy i po obnovení - se vše provádí odznovu - od prvního řádku a nezústává nic zachováno (uloženo).\\
Pokud potřebuji využít data, která jsem získal v předchozím zobrazení stránky, musím je z původní stránky odeslat na stránku, kterou zobrazuji nyní (i když je to stejná stránka - soubor). Také si mohu uložit data mimo stránku - do PC nebo na server\\
Odeslat data mohu např. pomocí metody \uv{post} (využíváme ve formulářích). Uložit data mohu pomocí cookies, případně session.

\subsection{Proměnné}
Proměnné v kódu si můžeme představit jako pojmenované \uv{krabičky}, do kterých si můžeme v průběhu programu ukládat různé informace.\\
Veškeré proměnné v PHP musí začínat \$ dolarem.\\
PHP je \uv{loosely typed language}. Pro nás to znamená předevsím to, že nmusíme při vytváření proměnné zapisovat, jakého typu budou data, která do ni budeme ukládat. Dokonce se datový typ může v průběhu progrmau měnit - POZOR na to!