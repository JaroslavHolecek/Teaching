\section{Užitečné}


\subsection{Proměnná \$\_SERVER}
Proměnná \uv{\$\_SERVER} je globální proměnná (tedy k ní máme přístup kdekoliv v programu), ze které můžeme získat spoustu užitečných informací: \url{https://www.php.net/manual/en/reserved.variables.server.php}.

\subsubsection{'REQUEST\_METHOD'}
V \uv{\$\_SERVER['REQUEST\_METHOD']} lze zjistit, pomocí jaké metody jsme načetli stránku. Pro nás to bude obzlváště užitečné při zpracování dat z formuláře. Můžeme zjistit, zda jsme stránku zobrazili bez odeslání dat (a tedy nemáme co zpracovávat), nebo po odeslání dat přes formulář (a tedy můžeme tyto data používat). \\

\begin{minipage}[t]{.45\textwidth}
\begin{code}
\begin{minted}[linenos, escapeinside=!!]{html+php}
<html>
<body>

<form action="soubor_pro_zpracovani.php" method="post"> !\label{scl:request_method_post_set}!
Počet řádků: <input type="number" name="radky"><br>
Počet sloupců: <input type="number" name="sloupce"><br>
<input type="submit" value="Zobraz tabulku">
</form>

<?php
if ($_SERVER["REQUEST_METHOD"] == "POST"){ !\label{scl:request_method_post_check}!
	$zadane_radky = $_POST["radky"];	!\label{scl:request_method_post_check_ok_start}!
	$zadane_sloupce = $_POST["sloupce"];	!\label{scl:request_method_post_check_ok_end}!
}else{
	echo "Zatím jsi nic neposlal.<br>"; !\label{scl:request_method_post_check_ko_start}!
	echo "Zadej počet řádků a sloupců, klikni na tlačítko a uvidíš ty divy...";	!\label{scl:request_method_post_check_ko_end}!
}

?> 

</body>
</html> 
\end{minted}

\captionof{listing}{Metoda POST}
\label{code:php_form_post}
\end{code}
\end{minipage}
\begin{minipage}[t]{.45\textwidth}
\vspace{7cm}
\begin{enumerate}
\item[ř. \ref{scl:request_method_post_set}:] Nastavení metody \uv{POST}.
\item[ř. \ref{scl:request_method_post_check}] Kontrola, zda na stránku přišly data pomocí metody \uv{POST}.  
\item[ř. \ref{scl:request_method_post_check_ok_start}-\ref{scl:request_method_post_check_ok_end}:] Pokud byla použita metoda \uv{POST}, můžeme data zpracovat.
\item[ř. \ref{scl:request_method_post_check_ko_start}-\ref{scl:request_method_post_check_ko_end}:] Pokud nebyla použita metoda \uv{POST}, uděláme něco jiného.
\end{enumerate}
\end{minipage}

\subsection{Funkce}
\subsubsection{empty(), isset()}
Funkce \uv{empty(\$promenna)} vrátí hodnotu \uv{True}, pokud je proměnná, kterou testujeme (vložíme do kulatých závorek), \uv{prázdná}. \uv{Prázdná} znamená, že neexistuje, nebo má hodnotu:\\ \textbf{0}; \textbf{0.0}; \textbf{"0"}; \textbf{""}; \textbf{NULL}; \textbf{FALSE}; \textbf{array()}\\
Obdobným způsobem pracuje funkce \uv{isset(\$promenna)}. Ta naopak vrátí hodnotu \uv{True}, pokud proměnná \textbf{existuje} a je v ní uložena hodnota \textbf{ruzná od NULL}.

\begin{minipage}[t]{.45\textwidth}
\begin{code}
\begin{minted}[linenos, escapeinside=!!]{html+php}
<html>
<body>

<form action="soubor_pro_zpracovani.php" method="post"> !\label{scl:request_method_post_set}!
Počet řádků: <input type="number" name="radky"><br>
Počet sloupců: <input type="number" name="sloupce"><br>
<input type="submit" value="Zobraz tabulku">
</form>

<?php
if (empty($_POST["radky"])){ !\label{scl:empty_radky}!
	echo "Nezadal jsi počet řádků!";	!\label{scl:empty_radky_ok}!
}else{
	$zadane_radky = $_POST["radky"]; !\label{scl:empty_radky_ko}!
}

if (isset($_POST["sloupce"])){ !\label{scl:isset_sloupce}!
	$zadane_sloupce = $_POST["sloupce"]; !\label{scl:isset_sloupce_ok}!
}else{
	echo "Nezadal jsi počet sloupců!";	 !\label{scl:isset_sloupce_ko}!
}
?> 

</body>
</html> 
\end{minted}

\captionof{listing}{empty()}
\label{code:php_empty}
\end{code}
\end{minipage}
\begin{minipage}[t]{.45\textwidth}
\vspace{9cm}
\begin{enumerate}
\item[ř. \ref{scl:empty_radky}:] Kontrola, zda byly zadány řádky
\item[ř. \ref{scl:empty_radky_ok}:] Pokud nebyly zadány - proměnná je prázdná - vypíšeme hlášku.
\item[ř. \ref{scl:empty_radky_ko}:] Pokud byly zadány - proměnná není prázdná - můžeme ji použít.
\item[ř. \ref{scl:isset_sloupce}:] Kontrola, zda byly zadány sloupce
\item[ř. \ref{scl:isset_sloupce_ok}:] Pokud byly zadány - proměnná je nastavena - můžeme ji použít.
\item[ř. \ref{scl:isset_sloupce_ko}:] Pokud nebyly zadány - hodnota proměnné je NULL, nebo neexistuje - vypíšeme hlášku.
\end{enumerate}
\end{minipage}

\subsubsection{header()}
Je funkce, která nám umožňuje např, přesměrovat stránku na jinou adresu.  Případně zaslat jakékoliv jiné \uv{HTML header} informace.

\begin{minipage}[t]{.45\textwidth}
\begin{code}
\begin{minted}[linenos, escapeinside=!!]{html+php}
!\label{scl:header_first}!
<?php
header("Location: http://www.seznam.cz/"); !\label{scl:header_redirect}!
exit; !\label{scl:header_exit}!
?>

<html> !\label{scl:header_after}!
</html> 
\end{minted}

\captionof{listing}{header()}
\label{code:php_header}
\end{code}
\end{minipage}
\begin{minipage}[t]{.45\textwidth}
\begin{enumerate}
\item[ř. \ref{scl:header_first}:] Před funckí header() nesmí být žádné jiné html řádky - ani vytvořené přes php, ani přímo napsané
\item[ř. \ref{scl:header_redirect}:] Přesměrování na jinou adresu
\item[ř. \ref{scl:header_exit}:] Zajištění, že se dále jistě nic neprovede
\item[ř. \ref{scl:header_after}:] Tento kód již nebude mít efekt
\end{enumerate}
\end{minipage}

\subsubsection{include(), require()}
Velmi často se nám na stránkách některé části opakují. Např. menu je všude na našich stránkách stejné, hlavička, zápatí, seznam sponzorů a pod.\\
Abychom mohli napsat kód pro např. menu pouze jednou, využijeme funkce include() či require().\\
Opakující se část kódu zapíšeme pouze jednou do samostatného souboru a poté pomocí těchto funkcí \uv{překopírujeme} kód ze zadaného souboru do místa, kde voláme tyto funkce.\\
Rozdíl mezi include() a require() je pouze v tom, že pokud námi požadovaný soubor není nalezen, při použití include() se stránka dále normálně načte - poue v ní bude chybět kód ze souboru, který jsme vkládali. Při použití require se načítání stránky zastaví. 

\begin{minipage}[t]{.45\textwidth}
\begin{code}
\begin{minted}[linenos, escapeinside=!!]{html+php}
<html>
<?php
require("menu.html"); !\label{scl:require}!
include("podekovani.html"); !\label{scl:include}!
?>
</html> 
\end{minted}

\captionof{listing}{include(), require()}
\label{code:php_include_require}
\end{code}
\end{minipage}
\begin{minipage}[t]{.45\textwidth}
\begin{enumerate}
\item[ř. \ref{scl:require}:] Vloží kód (celý obsah) ze souboru menu.html.\\Pokud soubor neexistuje, načítání stránky se zastaví. 
\item[ř. \ref{scl:include}:] Vloží kód (celý obsah) ze souboru podekovani.html.\\Pokud soubor neexistuje, stránka se zobrazí (samozřejmě bez vloženého kódu).
\end{enumerate}
\end{minipage}





