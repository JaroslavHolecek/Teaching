\section{Databáze}
Zdrojové kódy převzaty z \url{https://www.w3schools.com/php/php_mysql_intro.asp}, kde najdete i další ukázky.\\
Pokud chceme, aby naše webové stránky zpracovávaly větší množství informací (např. více uživatelů, různé zboží v e-shopu) je vhodné tyto informace uložit v databázi.\\
Zde ve škole můžeme využít školní databázi na serveru \url{dbs.spskladno.cz} (MySQL)- pro přístup přes prohlížeč použijte \url{http://dbs.spskladno.cz/myadmin/} . Přístupové údaje získáte od svých vyučujících.\\
Možností jak pracovat s databází pomocí php je více a mimo jiné záleží na typu databáze. Zde si ukážeme jednu z možností.\\

\subsection{Přístupové údaje}
Pro připojení k databázi (přesněji k databázovému serveru) je potřeba zadat údaje:
\begin{enumerate}
\item[\textbf{adresa serveru}] ip, nebo url adresa - dbs.spskladno.cz
\item[\textbf{jméno}] získáte od vyučujících
\item[\textbf{heslo}] získáte od vyučujících
\end{enumerate}
pro připojení ke konkrétní databázi navíc zadáme název databáze:
\begin{enumerate} \setcounter{enumi}{4}
\item[\textbf{název databáze}] zde na školním databázovém serveru závisí na účtu, pod kterým se k serveru přihlásíme
\end{enumerate}

Pomocí výše uvedených údajů vytvoříme spojení našeho php skriptu s databází. Přes toto spojení můžeme zasílat do databáze požadavky pomocí jazyku SQL.\\

\begin{minipage}[t]{.45\textwidth}
\begin{code}
\begin{minted}[linenos, escapeinside=!!]{html+php}
<?php
$servername = "dbs.spskladno.cz"; !\label{scl:dbs_pristupove_udaje}!
$username = "dotaz_na_vyucujiciho";
$password = "dotaz_na_vyucujiciho";
$dbname = "vyuka__";

$conn = new mysqli($servername, $username, $password, $dbname); !\label{scl:dbs_spojeni_start}!


if ($conn->connect_error) { !\label{scl:dbs_spojeni_kontorola}!
    exit("Spojení se nezdařilo. Chyba: " . $conn->connect_error);
}
$mysqli->set_charset("utf8mb4"); !\label{scl:dbs_charset}!

$conn->close(); !\label{scl:dbs_spojeni_konec}!
?> 
\end{minted}

\captionof{listing}{Připojení k databázi}
\label{code:php_dbs_spojeni}
\end{code}
\end{minipage}
\begin{minipage}[t]{.45\textwidth}
\begin{enumerate}
\item[ř. \ref{scl:dbs_pristupove_udaje}:] Uložení přístupových údajů.
\item[ř. \ref{scl:dbs_spojeni_start}:] Vytvoření spojení 
\item[ř. \ref{scl:dbs_spojeni_kontorola}:] Kontrola, zda se spojení podařilo.\\
Pokud ne, ukončí se skript a vypíše se chybová hláška.
\vspace{2cm}
\item[ř. \ref{scl:dbs_charset}:] Nastavení kódování - abychom mohli psát háčky a čárky
\item[ř. \ref{scl:dbs_spojeni_konec}:] Ukončení spojení 
\end{enumerate}
\end{minipage} 

\subsection{Dotazy bez výsledku}
U některých dotazů do databáze nezískáváme žádný výsledek (např. vytvoření tabulky, vložení dat do tabulky). U takových dotazů můžeme zjistit, zda dotaz proběhl v pořádku, případně se něco pokazilo (např. nemáme přístup, špatně zapsaný dotaz) a můžeme vypsat chybovou hlášku, kterou nám zaslal zpět databázový server.\\

\begin{minipage}[t]{.45\textwidth}
\begin{code}
\begin{minted}[linenos, escapeinside=!!]{html+php}
<?php
$servername = "dbs.spskladno.cz"; 
$username = "dotaz_na_vyucujiciho";
$password = "dotaz_na_vyucujiciho";
$dbname = "vyuka__";

$conn = new mysqli($servername, $username, $password, $dbname); 

if ($conn->connect_error) { 
    exit("Spojení se nezdařilo. Chyba: " . $conn->connect_error);
}
$mysqli->set_charset("utf8mb4");

$sql_create = "CREATE TABLE Uzivatel ( 
id INT PRIMARY KEY,
jmeno TEXT NOT NULL,
heslo TEXT NOT NULL
)"; !\label{scl:dbs_dotaz_create}!

$sql_insert = "INSERT INTO Uzivatel (jmeno, heslo) VALUES ('jarda', '123'), ('adam', '456')"; !\label{scl:dbs_dotaz_insert}!

$result = $conn->query($sql_create); !\label{scl:dbs_dotaz_create_zaslani}!
if ($result === TRUE) {	!\label{scl:dbs_dotaz_create_vysledek}!
    echo "Tabulka vytvořena.";
} else {
    echo "Tabulku se nepodařilo vytvořit. Chyba: " . $conn->error;
}

$result = $conn->query($sql_insert); !\label{scl:dbs_dotaz_insert_zaslani}!
if ($result === TRUE) {	!\label{scl:dbs_dotaz_insert_vysledek}!
    echo "Data vložena.";
} else {
    echo "Data se nepodařilo vložit. Chyba: " . $conn->error;
}

$conn->close();
?> 
\end{minted}

\captionof{listing}{Tvorba tabulky a vložení dat}
\label{code:php_dbs_create_insert}
\end{code}
\end{minipage}
\begin{minipage}[t]{.45\textwidth}
\begin{enumerate}
\item[ř. \ref{scl:dbs_dotaz_create}, \ref{scl:dbs_dotaz_insert}:] Příprava dotazů pro vytvoření tabulky a vložení dvou uživatelů.\\
\textcolor{red}{Ve skutečné databázi se hesla nikdy neukládají v této \uv{otevřené} formě.}
\item[ř. \ref{scl:dbs_dotaz_create_zaslani}, \ref{scl:dbs_dotaz_insert_zaslani}:] Zaslání dotazu do databáze.
\vspace{3.5cm}
\item[ř. \ref{scl:dbs_dotaz_create_vysledek}, \ref{scl:dbs_dotaz_insert_vysledek}:] Kontrola, zda dotazy proběhly v pořádku.\\
Pokud ne, ukončí se skript a vypíše se chybová hláška.
\end{enumerate}
\end{minipage}


\subsection{Dotazy s výsledkem}
U některých dotazů získáváme z databáze informaci (proto databázi máme \dots).\\
Výsledkem dotazu může být velmi mnoho řádků (např. e-shop má v nabídce mnoho triček) a bylo by značně náročné, přenášet celý výsledek najednou.\\
V následující ukázce si můžeme představit, že po odeslání dotazu se v databázi \textbf{dočasně vytvoří} náš \textbf{výsledek} a do našeho php skriptu \textbf{se přenese ukazatel} na tyto data (nepřenáší se tedy data samotná, ale pouze informace, kde v databázi jsou). \textbf{Část dat} (typicky jeden řádek) \textbf{se přenese} až \textbf{po zavolání funkce} \uv{fetch\_assoc()}. Díky tomu můžeme z databáze stahovat jen tolik dat, kolik potřebujeme.\\
\textbf{Data} jsou po přenesení \textbf{uložena v asociativním poli}, kde \textbf{název} jednotlivých \textbf{položek} (klíč) je \textbf{název sloupečku} v databázi.\\

\begin{minipage}[t]{.45\textwidth}
\begin{code}
\begin{minted}[linenos, escapeinside=!!]{php}
<?php
$servername = "dbs.spskladno.cz"; 
$username = "dotaz_na_vyucujiciho";
$password = "dotaz_na_vyucujiciho";
$dbname = "vyuka__";

$conn = new mysqli($servername, $username, $password, $dbname); 

if ($conn->connect_error) { 
    exit("Spojení se nezdařilo. Chyba: " . $conn->connect_error);
}
$mysqli->set_charset("utf8mb4");

$sql_create = "CREATE TABLE Uzivatel ( 
id INT PRIMARY KEY,
jmeno TEXT NOT NULL,
heslo TEXT NOT NULL
)";

$sql_insert = "INSERT INTO Uzivatel (jmeno, heslo) VALUES ('jarda', '123'), ('adam', '456')"; 

$result = $conn->query($sql_create);
if ($result === TRUE) {
    echo "Tabulka vytvořena.";
} else {
    echo "Tabulku se nepodařilo vytvořit. Chyba: " . $conn->error;
}

$result = $conn->query($sql_insert); 
if ($result === TRUE) {	
    echo "Data vložena.";
} else {
    echo "Data se nepodařilo vložit. Chyba: " . $conn->error;
}

$sql_select = "SELECT id, jmeno, heslo FROM Uzivatel"; !\label{scl:dbs_dotaz_select}!
$result = $conn->query($sql_select);

if ($result->num_rows > 0) { !\label{scl:dbs_dotaz_select_pocet_r}!

  while($row = $result->fetch_assoc()) { !\label{scl:dbs_dotaz_select_fetch}!
    echo "id: ". $row["id"]. " - Jméno: ". $row["jmeno"]. " - Heslo: ". $row["heslo"]. "<br>"; !\label{scl:dbs_dotaz_select_read}!
    }
    
} else {
    echo "Ve výsledku nejsou žádné řádky.";
}

$conn->close();
?> 
\end{minted}

\captionof{listing}{Select - dotaz s výsledkem}
\label{code:php_dbs_select}
\end{code}
\end{minipage}
\begin{minipage}[t]{.45\textwidth}
\begin{enumerate}
\item[ř. \ref{scl:dbs_dotaz_select}:] Příprava dotazu pro získání dat.
\item[ř. \ref{scl:dbs_dotaz_select_pocet_r}:] Zjištění počtu řádků ve výsledku.
\vspace{3cm}
\item[ř. \ref{scl:dbs_dotaz_select_fetch}:] Stažení jednoho řádku z výsledku a kontrola, zda jsou v něm data (po stažení všech řádků již budou stažená data prázdná)\\
Toto se opakuje, dokud není stažený řádek prázdný (již ve výsledku žádný není).
\vspace{1.2cm}
\item[ř. \ref{scl:dbs_dotaz_select_read}:] Ze staženého řádku čteme data jako z asociativního pole. 
\end{enumerate}
\end{minipage}

