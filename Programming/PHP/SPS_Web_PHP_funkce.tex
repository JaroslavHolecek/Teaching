\section{Funkce}
Funkce (někdy také nazývané metody) slouží k významnému zjednodušení a zpřehlednění kódu.\\
Funkci si můžete představit jako \uv{krabičku}, která umí dělat něco užitečného a kterou mám uloženou v paměti. Tuto \uv{krabičku} můžu v programu použít kolikrát chci.\\
Často chceme stejný proces (několik příkazů) spustit na několika místech v programu. Funkce nám umožní tento proces (několik příkazů) zapsat pouze jednou - pojmenovat ho - a poté ho spustit jen zadáním jeho jména. Nemusíme tak stále dokola psát stejný kód na všech místech, kde ho chceme spustit.\\
Pokud někde v programu píšete \textbf{podruhé} stejnou část kódu, již je to chvíle, kdy je čas na \textbf{použití funkce}.\\

\vspace{1cm}
\textit{Pro ukázku použití funkcí vytvoříme funkce, které budou zdravit naše kamarády a učitele}

\subsection{Vytvoření funkce}
Předtím, než můžeme funkci používat, ji musíme samozřejmě vytvořit - definovat. Definicí funkce ji pouze uložíme do paměti. V paměti funkce čeká do té doby, dokud ji nezavoláme (nespustíme).\\ 
Funkci vytvoříme zapsáním klíčového slova \uv{\textbf{function}}. Za ním následuje \textbf{název funkce} - jak chceme funkci volat. Dále jsou \textbf{kulaté závorky} - později do závorek zapíšeme takzvané argumenty, ale i když žádné argumenty psát nechceme, kulaté závorky zde být musí. Tomuto prvnímu řádku se říká \textbf{hlavička funkce}. Následuje tělo funkce, které je zapsáno do složených závorek. Do složených závorek zapíšeme všechny příkazy, které chceme mít uvnítř funkce - tedy ty, které se provedou, až funkci zavoláme (spustíme). Těmto řádkům říkáme \uv{tělo funkce}.\\


\begin{minipage}[t]{.45\textwidth}
\begin{code}
\begin{minted}[linenos, escapeinside=!!]{php}
<?php
function pozdrav(){	!\label{scl:php_funkce_def}!
	echo("Nazdar"); !\label{scl:php_funkce_start}!
	echo("Nazdar");
	echo("Nazdar"); !\label{scl:php_funkce_end}!
}	

echo("Franta"); !\label{scl:php_funkce_dalsi_kod}!
pozdrav(); !\label{scl:php_funkce_pouziti_1}!

echo("Lojza");
pozdrav(); !\label{scl:php_funkce_pouziti_2}!

echo("Mařenka");
pozdrav(); !\label{scl:php_funkce_pouziti_3}!
?>
\end{minted}

\captionof{listing}{Definice funkce}
\label{code:php_funkce_definice}
\end{code}
\end{minipage}
\begin{minipage}[t]{.45\textwidth}
\begin{enumerate}
\item[ř. \ref{scl:php_funkce_def}:] Definice (vytvoření, připravení) funkce: Klíčové slovo \uv{function}, název funkce, kulaté závorky\\
Dále začne tělo funkce otevírací složenou závorkou
\item[ř. \ref{scl:php_funkce_start}-\ref{scl:php_funkce_end}:] Tyto řádky (příkazy) se provedou po zavolání funkce - jsou uvnitř funkce - jsou odsazené. Říkáme jim \uv{tělo funkce}\\
Funkce končí zavírací složenou závorkou
\item[ř. \ref{scl:php_funkce_dalsi_kod}:] Program pokračuje na dalším řádku - již nepatří do funkce.
\item[ř. \ref{scl:php_funkce_pouziti_1}, \ref{scl:php_funkce_pouziti_2}, \ref{scl:php_funkce_pouziti_3}:] Voláme funkci. Na všech těchto řádcích skočí program do volané funkce (tedy na řádek \ref{scl:php_funkce_def}) a provede všechny příkazy uvnitř funkce.
\end{enumerate}
\end{minipage}


\subsection{Argumenty}
U funkce sice chceme, aby prováděla stále stejné příkazy, ale byly bychom rádi, aby uměla tyto stejné příkazy provést na různých datech (vstupech). Například pozdravit společně se jménem - a toto jméno bude pokaždé jiné (podle toho, koho zrovna zadravíme).\\
Abychom mohli dostat nějakou informaci (třeba jméno) dovnitř do funkce, použijeme argumenty funkce.\\

\begin{minipage}[t]{.45\textwidth}
\begin{code}
\begin{minted}[linenos, escapeinside=!!]{php}
<?php
function pozdrav($jmeno){	!\label{scl:php_funkce_def_arg}!
	echo("Nazdar $jmeno");
	echo("Nazdar $jmeno");
	echo("Nazdar $jmeno");
}
	
	
pozdrav("Franta"); !\label{scl:php_funkce_pouziti_arg_1}!

pozdrav("Lojza"); !\label{scl:php_funkce_pouziti_arg_2}!

pozdrav("Mařenka"); !\label{scl:php_funkce_pouziti_arg_3}!
?>
\end{minted}

\captionof{listing}{Funkce s argumentem}
\label{code:php_funkce_arg}
\end{code}
\end{minipage}
\begin{minipage}[t]{.45\textwidth}
\begin{enumerate}
\item[ř. \ref{scl:php_funkce_def_arg}:] Definice funkce s tím, že jí při volání předáme jeden argument - jméno, které má pozdravit: Argumenty (zde je pouze jeden, ale může jich být více) píšeme do kulatých závorek.
\item[ř. \ref{scl:php_funkce_pouziti_arg_1}, \ref{scl:php_funkce_pouziti_arg_2}, \ref{scl:php_funkce_pouziti_arg_3}:] Protože jsem vytvořili funkci s argumentem - musíme jí nějakou hodnotu tohoto argumentu předat.
\end{enumerate}
\end{minipage}

Argumentů můžeme funkci předat více, mohou být jakýchkoliv datových typů a mohou se ve funkci použít libovolně - tedy může být jeden argument string, druhý integer, atd. Více argumentů píšeme do kulatých závorek a \textbf{odělujeme je čárkou}.

\vspace{1cm}
\textit{Upravíme funkci tak, abychom jí mohli říct (předat argument), kolikrát má daného člověka pozdravit}\\

\begin{minipage}[t]{.45\textwidth}
\begin{code}
\begin{minted}[linenos, escapeinside=!!]{php}
<?php
function pozdrav($jmeno, $pocet){	!\label{scl:php_funkce_def_2arg}!
	for($i = 0; $i < $pocet; $i++){
		echo("Po $i: Nazdar $jmeno<br>");
	}
}
	
	
pozdrav("Franta", 1); !\label{scl:php_funkce_pouziti_2arg_1}!
pozdrav("Lojza", 3); !\label{scl:php_funkce_pouziti_2arg_2}!
pozdrav("Marenka", 10); !\label{scl:php_funkce_pouziti_2arg_3}!
\end{minted}

\captionof{listing}{Funkce s dvěma argumenty}
\label{code:php_funkce_2arg}
\end{code}
\end{minipage}
\begin{minipage}[t]{.45\textwidth}
\begin{enumerate}
\vspace{1.5cm}
\item[ř. \ref{scl:php_funkce_def}:] Definice funkce s dvěma argumenty.
\item[ř. \ref{scl:php_funkce_pouziti_2arg_1}, \ref{scl:php_funkce_pouziti_2arg_2}, \ref{scl:php_funkce_pouziti_2arg_3}:] Protože jsme vytvořili funkci s dvěma argumenty - musíme jí při volání předat dvě hodnoty.
\end{enumerate}
\end{minipage}

\subsection{Návratová hodnota - return}
Funkce mohou také spočítat výsledek (tak, jak znáte z matematiky - např. funkce 2x+1 spočítá pro vstup 1 výsledek 3, pro vstup 4 výsledek 9, pro vstup 7 výsledek 15 atd.).\\
Často u funkce chceme, aby nám výsledek, který spočítá, takzvaně \uv{vrátila} na místo (řádek) programu, odkud jsme funkci zavolali. V tomto místě (kde jsme funkci zavolali) typicky výsledek použijeme k dalším výpočtům, ale můžem s ním dělat cokoli chceme (nic, uložit ho, dále s ním počítat).\\
Že je již výpočet u konce (došli jsme k výsledku, který chceme vrátit) a chceme tedy funkci ukončit a vrátit výsledek, zepíšeme ve funkci pomocí klíčového slova \uv{return}. Po tomto klíčovém slově se již ve funkci neprovedou žádné příkazy. Pokud za slovo \uv{return} zapíšeme co má funkce vrátit, přenese se tato hodnota do místa, odkud jsme funkci zavolali.\\

\begin{minipage}[t]{.45\textwidth}
\begin{code}
\begin{minted}[linenos, escapeinside=!!]{php}
<?php
function pozdrav($jmeno, $pocet){	!
	for($i = 0; $i < $pocet; $i++){
		echo("Po $i: Nazdar $jmeno<br>");
	}
	
	return strlen(jmeno); !\label{scl:php_funkce_ret}!
	echo("Toto se jiz nevypise :-( ");
}	
	
	
$delka_jmena_1 = pozdrav("Jan", 1); !\label{scl:php_funkce_ret_uloz}!
echo("$delka_jmena_1");

echo(pozdrav("Vladislav", 3)); !\label{scl:php_funkce_ret_vypis}!

$delka_jmena_2 = pozdrav("Cecilie", 10); 

$soucet_delky_jmen = $delka_jmena_1 + $delka_jmena_2; !\label{scl:php_funkce_ret_pouzij}!
echo("Pocet znaku na pozvance: $soucet_delky_jmen")
?>
\end{minted}

\captionof{listing}{Funkce s return}
\label{code:php_funkce_return}
\end{code}
\end{minipage}
\begin{minipage}[t]{.45\textwidth}
\begin{enumerate}
\vspace{3.5cm}
\item[ř. \ref{scl:php_funkce_ret}:] Klíčové slovo \uv{return} a za ním hodnota, která se má vrátit - zde např. délka jména, které jsme funkci předali jako argument (např. u "Jan"  je vráceno 3, u "Cecilie" je vráceno 7).\\ Žádný další řádek už se neprovede.
\item[ř. \ref{scl:php_funkce_ret_uloz}:] Vrácenou hodnotu si můžeme uložit do proměnné (zde \uv{\$delka\_jmena\_1}).
\vspace{1cm}
\item[ř. \ref{scl:php_funkce_ret_vypis}:] Vrácenou hodnotu můžeme také přímo použít (zde zobrazit na stránce).
\item[ř. \ref{scl:php_funkce_ret_pouzij}:] Uložené hodnoty můžeme samozřejmě kdykoliv použít.
\end{enumerate}
\end{minipage}




